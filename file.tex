\documentclass[12pt]{article}

\usepackage{fontspec}
\usepackage{polyglossia}
\setmainlanguage{armenian}
\setotherlanguage{english}

% Set a Unicode Armenian font
\newfontfamily\armenianfont{Noto Serif Armenian}

\begin{document}

\title{Վերջավոր վարիացիայի ֆունկցիաներ}
\author{}
\date{}
\maketitle

\section{Վերջավոր վարիացիայի ֆունկցիայի սահմանում}
Հաշվի առնենք \(f(x)\) ֆունկցիան\(,\) որը սահմանված է \([a,b]\) փակ միջակայքում։ Բաժանենք \([a,b]\) հատվածը մի տրոհմամբ՝
\[
a = x_0 < x_1 < \dots < x_n = b.
\]
Տրված տրոհման համար կազմենք գումարը
\begin{equation}\label{eq:var-sum}
\sum_{i=0}^{n-1} \bigl|f(x_{i+1}) - f(x_i)\bigr|.
\end{equation}
Եթե գոյություն ունի որոշակի \(M\) կոնստանտ այնպիսի, որ անկախ տրված տրոհումից
\[
\sum_{i=0}^{n-1} \bigl|f(x_{i+1}) - f(x_i)\bigr| \le M,
\]
ապա ասում են\(,\) որ \(f\) ֆունկցիան \([a,b]\) միջակայքում ունի վերջավոր վարիացիա\(։\) Այդ դեպքում գումարների մեծագույն վերին եզրը կոչվում է \(f\)-ի \emph{լրիվ վարիացիա} \([a,b]\)-ում և նշվում է \(V_a^b(f)։\) Տրված միջակայքը բաժանելով ավելի խիստ\(,\) գումարների արժեքը կարող է այնքան մոտենալ \(V_a^b(f)\)-ին\(,\) որքան ցանկանանք\(։\)

Վերջավոր վարիացիայի հասկացությունը ընդարձակվում է նաև անվերջ միջակայքների վրա։ Ասում են\(,\) որ \( f(x) \)-ը ունի վերջավոր վարիացիա \( [a, +\infty) \) միջակայքում\(,\) եթե այն ունի վերջավոր վարիացիա յուրաքանչյուր \( [a, A] \) ենթատարածքում \( (A > a) \) և
\[
\sup_{A > a} V_a^A(f) < \infty.
\]
Այսինքն՝ բոլոր \( [a, A] \) ենթատարածքներում վարիացիան սահմանափակ է և չի մեծանում անսահմանության վրա։ Նշենք, որ այս սահմանման մեջ \( f \)-ի անընդհատությունը որևէ դեր չի խաղում. \( f \) կարող է չլինել անընդհատ և ունենալ վերջավոր վարիացիա։


Օրինակ՝ ցանկացած մոնոտոն ֆունկցիա ունի վերջավոր վարիացիա\(:\) Եթե \(f\) մոնոտոն աճող է \([a,b]-\)ում\(,\) ապա
\[
\sum_{i=0}^{n-1} |f(x_{i+1})-f(x_i)| = f(b)-f(a),
\]
և \(V_a^b(f)=f(b)-f(a)<\infty\) Եթե \([\)\(a,+\)\(\infty\)\(]\) հատվածում \(f\) մոնոտոն աճում է և գոյություն ունի \(f(+\infty)=\lim_{x\to+\infty}f(x)\)\(,\) ապա \(V_a^{+\infty}(f)=f(+\infty)-f(a)<\infty\)\(:\)

\section{Վարիացիայի ֆունկցիաների դասեր}
\begin{description}
\item\(1.\) Եթե \(f\) ֆունկցիան  \([a,b]\) -ն բաժանվում է վերջավոր թվով ենթատարածքների՝ \(a=a_0<a_1<\dots<a_m=b\)\(,\) որոնց յուրաքանչյուրի համար \(f֊\)ը մոնոտոն է\(,\) ապա \(f֊\)ը ունի վերջավոր վարիացիա \([a,b]-\)ում։ \\
Ապացույց\(:\) Անկախ տրոհման համար կարող ենք ընդգրկել բոլոր \(a_k\) բաժանման կետերը\(,\) ապա ամբողջ գումարը բաժանել հատվածների\(:\) Յուրաքանչյուր \([a_k,a_{k+1}]\) հատվածում \(f\)-ի փոփոխության գումարը \(\le |f(a_{k+1})-f(a_k)|\) է \(,\) ուստի
\[
\sum_{i=0}^{n-1}|f(x_{i+1})-f(x_i)| \le \sum_{k=0}^{m-1}|f(a_{k+1})-f(a_k)|<\infty.
\]
\item\(2.\) Եթե գոյություն ունի \(constant\) \(L,\) որ
\[
\bigl|f(\bar{x}) - f(x)\bigr| \leq L\, |\bar{x} - x| \quad  x, \bar{x} \in [a, b].
\]
\((\)Լիպշիցի պայման\()\) ապա \(f\)-ը ունի վերջավոր վարիացիա \([a,b]\)-ում։
Սա ակնհայտ է\(,\) քանի որ յուրաքանչյուր ենթատրոհման վրա
\[
|f(x_{i+1})-f(x_i)| \le L|x_{i+1}-x_i|
\]
և գումարելով ստացվում է \(\sum |f(x_{i+1})-f(x_i)| \le L(b-a)<\infty\)\(.\)
\item\(3.\) Եթե \(f\)-ը \([a,b]\) հատվածում շարունակական է և \(|f'(x)| \le L\)\(,\) ապա \(f\)-ը ունի վերջավոր վարիացիա\(,\) քանի որ Լանգրաժի թեորեմով
\[
|f(x_{i+1})-f(x_i)| \le L|x_{i+1}-x_i|,
\]
հետևաբար \(\sum |f(x_{i+1})-f(x_i)| \le L(b-a)<\infty\)\(.\)
\item\(4.\) Թող \(\phi(t)\) \([a,b]\)-ում բացարձակ արժեքով ինտեգրալն է\(,\) և \(f(x) = \int_a^x \phi(t)\,dt\) Այդ դեպքում \(f\)-ը ունի վերջավոր վարիացիա \([a,b]\)-ում, որովհետև
\[
|f(x_{i+1})-f(x_i)| = \Bigl|\int_{x_i}^{x_{i+1}}\phi(t)\,dt\Bigr| \le \int_{x_i}^{x_{i+1}}|\phi(t)|\,dt.
\]
Այսպիսով
\(\sum_{i=0}^{n-1}|f(x_{i+1})-f(x_i)| \le \int_a^b|\phi(t)|\,dt<\infty\).
\end{description}

\section{Վարիացիայի ֆունկցիաների հատկություններ}
\begin{description}
\item \(1.\) Եթե \(f\) ունի վերջավոր վարիացիա  \([a,b]\)-ում \(,\) ապա \(f\) սահմանափակ է այդ հատվածում։
\\
Ապացույց \(:\) Ըստ ցանկացած \(x\in[a,b]\) համար
\(\sum_{i=0}^{n-1}|f(x_{i+1})-f(x_i)| \ge |f(x)-f(a)|\),
սակայն \(\sum|f(x_{i+1})-f(x_i)| \le V_a^b(f)\) հետևաբար \(|f(x)-f(a)|\le V_a^b(f)\) և \(|f(x)|\le |f(a)|+V_a^b(f) < \infty\)։
\item\(2.\) Եթե \(f\) և \(g\) ունեն վերջավոր վարիացիա \([a, b]\)-ում\(,\) ապա նրանց գումարը \(f+g\), տարբերությունը \(f-g\) և արտադրյալը \(f\cdot g\) նույնպես ունեն վերջավոր վարիացիա։
\\
Ապացույց\(:\) Գումարի դեպքում
\[
|(f+g)(x_{i+1})-(f+g)(x_i)| \le |f(x_{i+1})-f(x_i)| + |g(x_{i+1})-g(x_i)|,
\]
և գումարելով ստացվում է \(V_a^b(f+g)\le V_a^b(f)+V_a^b(g)\)\(.\) Նույն ձևով \(V(f-g)\le V(f)+V(g)\). Արտադրյալն համար, եթե \(|f|\le K, |g|\le L\)\(,\) ապա
\[
|f(x_{i+1})g(x_{i+1}) - f(x_i)g(x_i)| \le K|g(x_{i+1})-g(x_i)| + L|f(x_{i+1})-f(x_i)|,
\]
և \(\sum |f(x_{i+1})g(x_{i+1}) - f(x_i)g(x_i)| \le K V_a^b(g) + L V_a^b(f) < \infty\)\(.\)
\item\(3.\) Եթե \(f\) և \(g\) ունեն վերջավոր վարիացիա \([a, b]\)-ում և \(\min_{[a,b]}|g(x)|\ge\sigma>0\)\(,\) ապա \(\frac{f}{g}\) ևս ունի վերջավոր վարիացիա\(:\)
\\
Ապացույց\(:\) Ցույց տալ \(\frac{1}{g(x)}\)-ը\(,\)
\[
\left|\frac{1}{g(x_{i+1})}-\frac{1}{g(x_i)}\right| = \frac{|g(x_{i+1})-g(x_i)|}{|g(x_{i+1})g(x_i)|} \le \frac{1}{\sigma^2}|g(x_{i+1})-g(x_i)|,
\]
և հետևաբար \(V_a^b(1/g)\le \sigma^{-2}V_a^b(g)<\infty\) Ուրեմն \(\frac{f}{g}=f\cdot\frac1g\) արտադրյալը նույնպես ունի վերջավոր վարիացիա։
\(4\) Թող \(a<c<b\). Հետևաբար \(f\)–ի վարիացիան \([a,b]\)-ում վերջավոր է եթե և միայն եթե այն վերջավոր է \([a,c]\) և \([c, b]\) ենթատարածքներում։ Արդյունքում \(V_a^b(f)=V_a^c(f)+V_c^b(f)\)\(.\)
Ըստ անհրաժեշտության դա սահմանափակ է\(,\) իսկ հակառակը՝ եթե \(V_a^c(f)<\infty\) և \(V_c^b(f)<\infty\), ապա ցանկացած \([a,b]\)-ի ենթատրոհման համար
\[
\sum_{i=0}^{n-1}|f(x_{i+1})-f(x_i)| \le V_a^c(f) + V_c^b(f)<\infty,
\]
ու \(V_a^b(f)\) վերջավոր է։
\end{description}

\section{Վերջավոր վարիացիա ունեցող ֆունկցիաներ և անընդհատություն}
\begin{description}
\item\([9°.]\) Եթե \(f\) վերջավոր վարիացիա ունի \([a,b]\)-ում և \(f\) անընդհատ է \(x_0\) կետում\(,\) ապա \(g(x)=V_a^x(f)\) (լրիվ վարիացիա \(x\)-ին) նույնպես անընդհատ է \(x_0\)-ում\(:\) \(f\)-ի անընդհատությունը ժառանգվում է \(g\)-ին։
\item\([10°.]\) Եթե \(f\) վերջավոր վարիացիա ունեցող և անընդհատ ֆունկցիա է \([a,b]-\)ում\(,\) ապա այն ներկայացվում է երկու մոնոտոն աճող անընդհատ ֆունկցիաների տարբերություն՝ \(f(x) = g(x)-h(x)\), որտեղ \(g\) և \(h\) մոնոտոն աճող և անընդհատ են։
\item\([11°.]\) Եթե \(f\) անընդհատ է \([a,b]-\)ում\(,\) ապա \(V_a^b(f)\)-ը հավասարվում է բաժանումների առավելագույն երկարության \(\|\Delta\|\to0\) սահմանին\(։\)
\[
V_a^b(f) = \lim_{\|\Delta\|\to0} \sum_{i=0}^{n-1}\bigl|f(x_{i+1}) - f(x_i)\bigr|.
\]
\end{description}

\section{Ուղղելի կորեր}
Վերջավոր վարիացիայի ֆունկցիաների հասկացությունը կիրառվում է կոր գծի ուղղելիության (ուղղելի լինել) հարցում: Կորը, որի վրա առաջին անգամ անդրադարձել է Ժորդանը, արտահայտվում է պարամետրային ձևով
\begin{equation}\label{eq:curve-param}
x = \varphi(t),\quad y = \psi(t),\qquad t_0 \le t \le T,
\end{equation}
որտեղ \(\varphi(t)\) և \(\psi(t)\) անընդհատ են, և կորի վրա կրկնվող հատվածներ չկան։ \(t_0 < t_1 < \dots < t_n=T\) բաժանումների արդյունքում կորը ծալում ենք օղակների հետևորդ, որի երկարությունը (նեոդոն) է
\[
\sum_{i=0}^{n-1} \sqrt{(\varphi(t_{i+1})-\varphi(t_i))^2 + (\psi(t_{i+1})-\psi(t_i))^2}.
\]
Կորի աղեղի երկարությունը ստացվում է այդ գումարների ճշգրիտ վերին եզրով։ Եթե այդ վերին եզրը վերջավոր է\(,\) կորը կոչվում է ուղղելի \((\)ուղղելի երկարություն ունի\():\) Հաջորդ թեորեմը տալիս է դրա համար անհրաժեշտ և բավարար պայմանը\(,\)

\textbf{Ժորդանի թեորեմը։} Կորի պարամետրային ներկայացումը \([t_0,T]\) միջակայքում ուղղելի է եթե և միայն եթե \(\varphi(t)\) և \(\psi(t)\)–ն ունեն վերջավոր վարիացիա \([t_0,T]-\)ում։

\textit{Անհրաժեշտություն\(:\)} Եթե կորն ուղղելի է և ունի \(L\) երկարություն\(,\) ապա ցանկացած ենթատրոհման \(\{t_i\}\) համար
\[
\sum_{i=0}^{n-1}|\varphi(t_{i+1})-\varphi(t_i)|
\le \sum_{i=0}^{n-1}\sqrt{(\varphi(t_{i+1})-\varphi(t_i))^2+(\psi(t_{i+1})-\psi(t_i))^2} \le L.
\]
Հետևաբար \(V_{t_0}^{T}(\varphi)\le L<\infty\) և \(V_{t_0}^{T}(\psi)\le L<\infty\): այսպիսով \(\varphi\) և \(\psi\) ունեն վերջավոր վարիացիա։

\textit{Բավարարություն\(:\)} Ենթադրենք\(,\) որ \(\varphi\) և \(\psi\) ունեն վերջավոր վարիացիա [\(t_0,T\)]–ում։ Հետևաբար ցանկացած ենթատրոհման համար
\[
\sqrt{(\varphi(t_{i+1})-\varphi(t_i))^2+(\psi(t_{i+1})-\psi(t_i))^2} \le
|\varphi(t_{i+1})-\varphi(t_i)| + |\psi(t_{i+1})-\psi(t_i)|.
\]
Այնուհետև
\[
\sum_{i=0}^{n-1}\sqrt{\dots} \le \sum_{i=0}^{n-1}|\varphi(t_{i+1})-\varphi(t_i)| + \sum_{i=0}^{n-1}|\psi(t_{i+1})-\psi(t_i)|
\le V_{t_0}^T(\varphi) + V_{t_0}^T(\psi) < \infty.
\]
Ուստի կորի երկարության վերին եզրը վերջավոր է\(,\) և \(K\) կորը ուղղելի է\(:\)

\end{document}
